\documentclass{article}
\usepackage{pythontex}
\usepackage{gensymb}
\usepackage{amssymb}
\usepackage{csvsimple}
\title{Determine Refractive Index of Material of a prism using Sodium Source}
\author{Preetpal Singh}
\date{\today}
\begin{document}
    \maketitle
\begin{pycode}
import pandas as pd
data = pd.read_csv('data1.csv')
	
\end{pycode}
\section{Aim}
To Determine Refractive Index of Material of a prism using Sodium Source
\section{Apparatus}
Spectrometer, prism, prism clamp, sodium vapour lamp, lens. etc.
\section{Procedure}
\begin{itemize}
    \item Focus Telescope on distant object.
    \item When focus is correct, start button is activated. Then click Start button.
    \item Switch on the light by clicking Switch On Light button.
    \item Focus the slit using Slit focus slider.
    \item Vernier is rotated in a way that light from collimator is incident on one face of prism and it refract the ray through second face.
    \item Telescope is adjusted to coincide slit image with cross wire.
    \item Vernier fine adjustment is used to make slit stationary for some moment in field of view.
    \item Telescope fine adjustment is used to coincide slit again with cross wire.
    \item Readings of 2 verniers are taken.
    \item Click on REMOVE PRISM.
    \item Direct ray is get into telescope(by moving it) from collimator.
    \item Slit is coincided with crosswire, readings are taken.
    \item Difference of these readings give angle of minimum deviation.
\end{itemize}

\section{Precautions}
\begin{itemize}
    \item SLit should be as narrow as possible. 
    \item Vernier numbering should remain fixed throughout the experiment.
    \item Prism position should be maintained properly.
    \item Fine adjustment of telescope must be used in each case.
    
\end{itemize}
\newpage
\section{Observations}
\subsection{Least Count of Spectrometer} 

$27 MSD = 30 VSD$
\newline
$1 VSD$ = $\frac{27}{30}$MSD
\newline
Least count = $1 MSD - 1 VSD$
\newline
= $1 MSD$ - $\frac{27}{30}$$MSD$ = $\frac{3}{30}$MSD$

 On main scale
\newline
20 divisions = 10$\degree$ 
\newline
1 division = $(1/2)\degree$ = 1 MSD
\newline 
\therefore L.C = (\frac{3}{30}) \times (\frac{1}{2})\degree = (\frac{1}{20})\degree
\newline
(\frac{1}{20} \times 60)' = 3'

\subsection{Angle of Deviation}
\csvautobooktabular{refractive_index_obs.csv}
\subsubsection{Take observation for angle of deviation for various angles of incidence i}





\subsubsection{Draw graph of delta vs i} 
\section{Calculations and Error Analysis}
\subsubsection{Angle of minimum deviation from graph}
\subsubsection{Error in angle of minimum deviation}
\section{Result and Discussion}
\section{Contribution of Team Mates}
List of contribution of each of the partner
Partner A:
\end{document}
